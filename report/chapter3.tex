\chapter{Experiment Design}
\centerline{\rule{149mm}{.02in}}
\vspace{2cm}

A series of experiments have been designed in order to objectively analyse the efficiency of Stratosphere and Hadoop. This section discusses the design of the experiments in regards to the aims of the project.

Experiments are designed in a modular fashion, enabling a subset of them to be completed whilst still providing relevant research results. This approach was taken to limit the risk that is introduced by the time constraints of the project. 

\section{Experiment Variables}
There are a series of factors which will be varied in the experiments, in order to provide a comprehensive answer to the research questions.

\subsection{Data Processing Tools}
The first factor that will be varied is the tool used to process the data. This relates to the first two research goals: providing a comparison of the MapReduce and PACT programming paradigms, and ascertaining how well Stratosphere can perform MapReduce tasks.

\subsubsection{Hadoop}
Hadoop is an open source implementation of the MapReduce paradigm, which has become the de-facto standard for big data processing \cite{qin2013reflection}. This makes it an ideal choice for comparing MapReduce and PACT.

Experiments will be run in Hadoop to provide an indication of the efficiency for the current state of the art in data processing. This provides a reference point to which alternative data processing technologies can be compared. 

\subsubsection{Stratosphere (PACT)}
Stratosphere is a tool which implements the PACT programming paradigm, which is a superset of MapReduce. This means there is a range of tasks that can be solved more expressively using a Directed Acyclic Graph and the range of extra operators that PACT provides.

Using all of the available features of PACT allows a complete comparison of Hadoop and Stratosphere. In real-world data processing scenarios, users of Stratosphere would be unlikely to restrict their programming options to just MapReduce, and would be likely to use other aspects of PACT as appropriate. Running experiments in this way allows a fairer comparison of the 2 tools, where all of their functionality is available.

\subsubsection{Stratosphere (MapReduce)}
One experiment will be run using just the Map and Reduce operations of Stratosphere. Whilst Stratosphere has a wider set of operations, solving data processing tasks using just Map and Reduce allows Stratosphere and Hadoop to be directly compared, aiming to answer the second research question. If Stratosphere can perform MapReduce tasks with comparable efficiency to Hadoop, it will indicate that Stratosphere is a more efficient tool for large-scale data processing in general, not just for those situations where a problem cannot be formulated in terms of MapReduce.

\subsection{Data Processing Task}
The second varying factor will be the data processing task that the tools have to complete. The tasks have been chosen to represent real-world issues as closely as possible, to give an indication of performance in scenarios where the tools are likely to be used, rather than having synthetic benchmarks such as word counting or Terasort. 

\subsubsection{Reverse Link Graph}
The first problem the data processing tools will have to solve is generating a Reverse Link Graph \cite{dean2008mapreduce}. In some situations, finding the transpose of a directed graph can be difficult. For example, when considering the Web Graph; it is relatively trivial to find the edges coming out of a vertex (as the hyperlinks on a page indicate its outgoing edges), but it is far more difficult to find incoming edges (pages which link to the current page). 

The Reverse Link Graph problem has been solved as it is a `classic' MapReduce problem. A `classic' MapReduce problem is a problem which can easily be formulated in terms of MapReduce. Classic MapReduce problems tend to be stateless, and do not depend on other parts of the data. This makes it easier to compute the solution in parallel, but means that many problems are not well suited to MapReduce (such as naturally iterative problems). An example of a classic MapReduce problem would be Word Counting, as each document can be processed in parallel. 

The Reverse Link Graph has been chosen in order to measure how effective both Hadoop and Stratosphere are at processing data using the MapReduce paradigm. This specifically relates to the second research question (``How well does Stratosphere perform MapReduce tasks?'').

Finding a reverse link graph of the web is a key component in many search engines \cite{page1999pagerank} where backlinks are used to indicate how important a page is, and can also be used to analyse a network of academic articles by treating references as edges \cite{garfield2002algorithmic}. 

\subsubsection{PageRank}
The second problem for the data processing tools to tackle is the PageRank algorithm. PageRank is a method of objectively and mechanically rating the importance of a web page, trying to measure human interest and the attention paid to the page \cite{page1999pagerank}. 

The PageRank algorithm is an example of the type of problem that MapReduce is traditionally cited as being poor at completing. It is an iterative algorithm which converges on a PageRank for each node. It requires information from adjacent nodes to compute a nodes PageRank at each step, which means that each step of the iteration must be performed sequentially, as part of a separate MapReduce pass.

By contrast, PageRank should use many of the additional features that the PACT programming model provides over MapReduce. The presence of iteration support and dedicated operations for Joins should test the additional operations that PACT provides in order to determine whether they yield tangible performance benefits.

Computing the PageRank of a graph of pages is a suitable test of how well the PACT programming model overcomes the issues of MapReduce, as it exercises many of the additional features. This will provide an answer to the first research question (``Does PACT overcome the inherent disadvantages of the MapReduce paradigm?'').

\subsection{Cluster Size}
The size of the cluster will be varied in order to give an indication of how well the data processing technologies perform at scale. This is a key factor in processing large scale data, as technologies which do not scale well as machines are added are unlikely to perform well in real-world scenarios, where the size of data may run into the Petabytes.

The elasticity of the Cloud will be required to scale the size of the cluster for different experiments. Specifically, experiments will be run using 2, 4, 6, 8 and 10 nodes. Scaling the size of the cluster will test both the scalability of the data processing tools, and the ability for the tools to harness the elasticity of the cloud, meeting the third research aim.

\section{Measurements}
Measurements will be taken from the experiments in order to give quantitative metrics which can be used for objective analysis.  

\subsection{Runtime}
Runtime is the length of time that a program (or data processing task) takes to execute. Runtime is one of the primary mechanisms for determining the efficiency of the data processing solutions. The time it takes to perform a task is a key element in large scale data processing, as the sheer size of data can cause operations to take a long period of time. 

\subsection{Scalability}
The Scalability of data processing tools is also a key metric in determining how efficient they are. The scalability of a system refers to how well it accommodates additional resources \cite{bondi2000characteristics}, such as additional machines. 

In the context of Data Processing tools such as Hadoop and Stratosphere, it is desirable for systems to be able to scale according to both the size of data, and the number of nodes in the cluster. A data processing tool should be able to handle any amount of data, providing that it has the correct amount of system resources. Indeed, this is what distinguishes tools like Hadoop from other data processing tools such as NumPy \cite{numpy} - the ability to analyse petabytes of data. A Data Processing system should also be able to scale according to the number of nodes added. There should be no upper limit to the number of nodes in a cluster, as this provides an important mechanism of adding the necessary resources to handle larger processing tasks. Adding a large number of nodes with lesser resources may be more cost efficient than adding fewer nodes which have comparatively larger system resources. This may be a reason why many users that engage in large scale data processing build clusters from commodity components \cite{taylor2010overview}.  

The specific type of scalability that will be considered in these experiments is Runtime Scalability. Intuitively, as nodes are added to the cluster the runtime for experiments should decrease as work can be partitioned out to a larger set of working nodes. In an ideal world, this would follow a linear progression; as the number of processors is doubled the runtime should be halved. This can be measured by taking the runtime of a task when the processor count doubles (so for 2, 4, 8, etc nodes) and comparing the results.

\subsection{Alternative Measurements}
Various alternative mechanisms could be used to determine the efficiency of the data processing tools. The scope of the project will be limited to measuring the runtime and runtime scalability, but some other measurements which could be used are as follows.

\subsubsection{Data Scalability}
It is important for clusters to be able to deal with large amounts of data in a reasonably efficient manner. A further test of scalability would be to test the cluster with various different datasets of different sizes. As with runtime scalability, this would ideally be linear, so that when the size of the data is doubled, runtime would also approximately double. The data processing tools may be of less use if the runtime increase is larger (for example, an exponential increase in runtime) as this would make running tasks on petabytes of data unreasonable.

Data scalability will not be measured in this project because of time and resource constraints. The virtual machines running the data processing cluster have relatively modest resources, and experiments have to run on 2 nodes. This limits the size of datasets. Acquiring and pre-processing large amounts of data also takes a significant amount of time, making it infeasible for a Final Year Project.

\subsubsection{Resource Utilisation}
The resources used on nodes can be measured to determine how efficient the data processing tool is. An efficient data processing tool could be defined as one which uses the resources of each node effectively, by utilising a high amount of memory and CPU capacity. This would indicate that the data processing tool is using each node to the highest capacity possible, rather than requiring extra nodes to be added for more resources to become available.

The utilisation of the network can also be used in determining efficiency, as a significant factor in how well a system scales is how much it communicates with other nodes. Larger amounts of communication overhead will increase the runtime of a system as it scales. Measuring the network overhead gives an indication of how much nodes communicate with each other, therefore giving an indication of communication overhead and scalability.

Resource Utilisation will not be measured in this project due to the nature of the cluster. Resources such as CPU are shared with other virtual machines on the same physical hardware, and the cluster does not have virtual networking enabled. This makes it difficult to get a reliable measure of resource utilisation without the results being influenced by external factors.

\subsubsection{Cost}
Another metric which can be used to compare data processing tools is cost efficiency. In situations where a cluster is being built using Cloud Computing services, running data processing jobs will cost on a per-usage basis. Depending on the runtime scalability, it may be more cost efficient to run a job on a larger number of nodes (so that it completes faster), versus running it on a smaller number of nodes (which will take longer, but cost less per hour). 

It may also be more cost efficient to create a private cluster of machines to run data processing jobs, rather than using a pay-per-usage cloud model.

Cost will not be used as an evaluation measure for this project, as it is mainly relevant in public cloud scenarios, and it would be difficult to measure the cost of creating a dedicated data processing cluster without very specific data processing requirements. 

\section{Hypothesis}
The project as a whole focuses on answering 3 core research questions. Rather than hypothesising the result to each individual experiment, there will be a prediction for each research question. Each experiment is used as a way of gathering information to answer one of the three core questions, and as such the results to individual experiments can be inferred from the broader descriptions.

\tocless\subsection{Does PACT overcome the inherent disadvantages of the MapReduce paradigm?}
In order to determine whether PACT overcomes the inherent disadvantages of MapReduce, both tools will solve the PageRank problem, a data processing task which exhibits many of the features generally considered to be a poor fit for MapReduce.

It is likely that PACT will demonstrate a clear superiority over MapReduce in these situations by having a smaller runtime, regardless of the number of nodes that are used to process the data. The presence of dedicated PACT operators for both iteration and delta-iteration will be a significant benefit for PACT, as PageRank is a naturally iterative algorithm. Additionally, PACT provides operators for joins, which are used throughout the PageRank algorithm. This will allow the Stratosphere runtime to optimise these steps accordingly.

In contrast, MapReduce lacks a dedicated method of iteration, meaning that any iterative algorithm must be implemented using multiple Map and Reduce steps. Additionally, any joins required by the algorithm will have to be implemented using additional Map and Reduce operations, causing multiple steps per iteration.

This will cause the PACT runtime to outperform the MapReduce runtime when running the PageRank experiment, indicating that it overcomes the inherent flaws in the MapReduce paradigm.

\tocless\subsection{How well does Stratosphere perform MapReduce tasks?}
The Reverse Link Graph problem will be used to determine how well Stratosphere performs MapReduce tasks. As a classic MapReduce problem it allows comparison with Hadoop, the current status quo of data processing tools.

In these situations Hadoop should be the clear winner in terms of performance. It is a simpler tool, designed for running MapReduce processes without the overhead associated with having other operations available. This allows the MapReduce runtime to be highly optimised to performing those tasks. As a research project, it is far more likely that the limited development resources of Stratosphere would be focussed towards optimising other operations, designed to be the source of an advantage over MapReduce. This may lead to a less efficient MapReduce implementation. 

Stratosphere is also a relatively immature project in comparison to Hadoop. It has existed for far less time, and therefore has less contributions. It is a less visible project than Hadoop, and has been used less in production. On this basis it can be reasoned that Stratosphere will have had less development attention towards optimising the Map and Reduce processes, and will therefore have inferior performance with these tasks.

\tocless\subsection{How do Hadoop and Stratosphere perform in a highly elastic Cloud Computing \\ environment?}
An important factor in how suitable data processing tools are to large scale data sets is how well they scale. This is particularly important in Cloud Computing environments. In order to evaluate how well the tools scale experiments will be run using a varying number of nodes, with the difference in runtime evaluated to determine how well the data processing tools respond to having additional resources available. 

Both Stratosphere and Hadoop (and in a broader sense, PACT and MapReduce) are based on stateless functions being distributed across nodes, evaluating data massively in parallel. Because of the nature of this model, communication overhead (the primary cause of parallel overhead) should be minimised during the primary processing time of the tools. It can therefore be expected that both Stratosphere and Hadoop will increase in a generally linear fashion.

It would be unrealistic to expect the tools to scale in a perfectly linear manner, as there is still communication required between the nodes. In particular, some operations can be made parallel more easily than others; map is a naturally parallel action, where the nature of reduce requires communication between at least 2 completed map operations. There will also be communication required at the start and end of each job, as input and output data must be read from, and written to, the file system. Finally, there will be overhead required to schedule tasks to nodes throughout the cluster. This process is centralised, with the master node determining which nodes will be responsible for processing different parts of the data at each stage. This computation is serial by necessity, and in contrast to most other processing tasks may increase in time as more nodes are added (as the scheduling problem will become progressively more complex). 

One major cause of scalability issues in parallel computing can be the underlying file system. Communicating to disk is expensive, and will be even more so if a node has to communicate over the network to get access to the file system. The file system implementation may therefore play a key role in determining the scalability of the system. An inferior file system implementation may lead to more network communication, significantly negatively impacting the runtime of a processing task. As both Stratosphere and Hadoop use the Hadoop Distributed File System, this discrepancy will not be an issue when assessing the relative scalability of the two systems.

Based on this analysis, it can be expected that Stratosphere and Hadoop will not show any major differences in scalability, with both scaling in a near-linear fashion.

\section{Conclusion}
A series of experiments have been designed to answer the key research questions determined at the start of the project. The experiments have been designed to test the performance of both Hadoop and Stratosphere in performing MapReduce and non-MapReduce tasks. The scalability of the systems will also be analysed. Specifically, both tools will compute the Reverse Link Graph and PageRank of a set of data. 