\chapter{Introduction}
\label{CHAP_FIRST}
\centerline{\rule{149mm}{.02in}}
\vspace{2cm}
\section{Project Aim}
Large scale data processing is an emerging trend in Computing, with benefits to both academia and industry. The aim of this project is to investigate the challenges of large scale parallel data processing, discuss the benefits of Cloud Computing and investigate how the efficiency of large scale data processing may be improved. 

This project will aim to provide a reasoned and objective evaluation of the efficiency of data processing techniques, whilst comparing current state of the art tools and technologies with newer research in the field. 

Specifically, the data processing tools Hadoop and Nephele will be compared, allowing comparison of the individual tools, and a wider discussion of the pros and cons of the MapReduce and PACT programming paradigms. 

\subsection{Research Questions}
The project aims to answer the following questions:

\begin{itemize}
	\item Does PACT overcome the inherent disadvantages of the MapReduce paradigm?
	\item How well does Nephele perform MapReduce tasks?
	\item How do Hadoop and Nephele perform in a highly elastic Cloud Computing environment?
\end{itemize}  

\section{Objectives}
The objectives of the project are:

\begin{itemize}
	\item Investigate Cloud Computing, Big Data and other relevant background trends
	\item Design a series of experiments to determine the efficiency of data processing technologies
	\item Implement the experiments for both Hadoop and Nephele
	\item Evaluate the experiment results to draw meaningful conclusions to the research questions
\end{itemize}

\section{Methodology}
The project will be broken into three main phases. The order to the phases provides a structured approach to the project, and completion of each phase ensures the project is completed in a methodical manner. The \textbf{Background Research} section will provide a comprehensive literature review to provide context to the project. This will be followed by a phase of \textbf{Experimentation}, which will aim to test the necessary factors of Hadoop and Nephele to answer the research questions. Finally, the \textbf{Evaluation} phase will evaluate the results of the experiments and the project as a whole.

The phases of the project will be performed according to Agile principles. Work will be carried out in weekly sprints, with clearly defined targets for each sprint. Supervisor meetings will be used as an opportunity to reflect on the achievements and shortcomings of the previous week, and to set targets for the upcoming sprint.

\subsection{Background Research}
The background research will aim to provide context for the project by surveying the current research landscape. It will summarise the fundamental concepts underlying the project and give a strong theoretical foundation for the rest of the project.

The first section of the background research will focus on Cloud Computing. It will explore the principles of the cloud, the service and deployment models and the technologies which make the cloud possible. It will also look at the unique advantages that the cloud provides in data processing scenarios.

The second segment of the background research will look in to Big Data and what challenges it introduces. An overview will be given of tools and techniques which have been developed to process Big Data, including Hadoop and Nephele.

\subsection{Experimentation}
Experiments will be designed to test the efficiency of data processing tools in relevant situations. This will require identifying relevant areas of interest, and deriving scenarios which will test the efficiency of Hadoop and Nephele. The experiments should be designed to take into account the various different problems that data processing tools are required to solve and to provide findings for the research questions.

As part of the experiment design, a hypothesis will be produced.

The experiments will be implemented and executed, in order to provide results which can be used to answer the research questions.

\subsection{Evaluation}
The project can be evaluated on various factors.

A technical evaluation will be carried out to analyse the results of the experiments. It will compare the efficiency of Hadoop and Nephele, referring to data obtained in the experimentation phase. The results will be compared to existing research (obtained during the background research) and the hypothesis, and any discrepancies will be discussed. 

An evaluation will also be carried out to determine whether the aims and objectives of the project have been met. This will involve determining whether the research questions have been answered in a satisfactory manner, and determining the relevance of the project in regards to existing literature and how it contributes to the research landscape. Any future scope for research will be identified. 

The methodology and management of the project will also be evaluated to identify areas of improvement in project management. This evaluation will be carried out in the personal reflection.

\section{Schedule}
The project schedule will be created around a set of fixed milestones in the project process. Work will be allocated in weekly sprints to ensure that the relevant work is completed in time for each milestone. Work will be prioritised so that scope can be cut to meet a milestone if required.

\subsection{Milestones}
There are a series of milestones which must be met throughout the project. This provides a framework for the project plan, as necessary tasks must be completed before the milestones.

\begin{enumerate}
	\item Submission of Aims \& Objectives

	The Aims \& Objectives of the project must be decided, which is crucial in setting the scope and direction for the rest of the project.

	\item Presentation to the Distributed Systems and Services Group

	The presentation to the Distributed Systems and Services group provides an opportunity to get early feedback on the project, and present much of what will be present in the mid year report. In order to present the project, a sufficient amount of content must be prepared. By this point a substantial amount of background research should have been carried out, and early ideas about experimental design should be formed.

	\item Mid-Year Project Report

	The Mid-Year Project Report is a chance to get feedback from the assessor about the direction and progress of the project. By this stage the background research should be completed, and should demonstrate that necessary steps have been taken to understand the problem. The Aims \& Objectives should be clearly defined, with a set of deliverables, a methodology, and a schedule. At this point experimental design will not have been completed, but the Mid-Year Report should give indications about the type of experiments that will be executed and why they were chosen. Finally, evidence of basic implementation should be shown, to demonstrate knowledge of the tools required for experiment implementation. 

	\item Progress Meeting

	The progress meeting occurs in the late stages of the project, so at this point most of the work should be completed. Experiments should have been implemented and performed, and code should be available for inspection.

	\item Final Report Submission
	The final report submission marks a point where all work on the project has to be complete. 
\end{enumerate}

\section{Deliverables}
The project will deliver 2 things.

\begin{enumerate}
	\item An evaluation of the efficiency of Hadoop and Nephele in various data processing situations
	\item Code for experiments, designed to test the efficiency of Hadoop and Nephele
\end{enumerate}