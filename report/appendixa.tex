\chapter{Personal Reflection}
\centerline{\rule{149mm}{.02in}}
\vspace{2cm}

At the beginning of my degree I found it difficult to imagine dedicating such a large amount of time to one topic. 

I feel one of the key factors in finishing the project successfully was following an Agile approach. Reviewing the work completed on a weekly basis helped make more realistic estimates for future work, and led to more realistic scheduling. In order to facilitate this, I would recommend other students to provide a relatively loose schedule in the early stages of the project. It is likely that the (feasible) requirements for the project will become more clear as the project progresses, and having a flexible schedule allows for changes as things become clearer. Having a loose schedule also allows for adjustments when things go wrong. Inevitably tasks will take longer than expected and work will appear which was not anticipated. 

The Final Year Project provides very strict milestones in terms of the deadlines imposed on students. My experiment design was modular in nature, so certain experiments could be cut or simplified when there was not enough time to implement the original plan. The ability to cut scope whilst still meeting the minimum requirements is essential in a project of this nature, with very strict, immovable deadlines. 

Maintaining this methodology relies on the student having discipline to set clear goals each `sprint', and to review the progress towards those goals at the end in order to accurately ascertain what amount of work is feasible for each time period. For the Final Year Project I found that a week was naturally suited to being the length of a sprint, as supervisor meetings provided an opportunity to reflect on progress and set goals. It is much easier to set realistic expectations in the short term, rather than over the length of the whole project.

Taking the time to set out a strong set of Aims \& Objectives at the beginning of the project was something which served me well, as it provided a clear end goal for the project. While the method of getting there changed, having a core direction to aim toward was helpful.

From a technical perspective I would encourage students undertaking Big Data projects to be wary of the length of time it takes to correctly configure Hadoop and similar tools. They are far more complex than they initially appear; this was the cause of the largest deviation from my project schedule. Similarly, I found several problems with libraries and frameworks. In particular, working with a beta version of Stratosphere caused significant problems at points. I would encourage future students to work on stable branches of software whenever possible, and to strongly consider whether the move to a less stable version is worth the additional features.

Using version control helped greatly throughout the project, allowing me to experiment with the experiment code without worrying about losing any prior, working code. It also made it much simpler to work on any machine, and ensured that the project was well backed up throughout. 

I would strongly encourage future students to take a project in an area of interest, and to choose a topic which will challenge them. The Final Year Project provides an excellent opportunity to learn new topics and work in interesting areas in much more depth than the rest of the degree program.
\todo{Scala}